%%%%%%%%%%%%%%%%%%%%%%%%%%%%%%%%%%%%%%%%
% INTRODUCTION
%%%%%%%%%%%%%%%%%%%%%%%%%%%%%%%%%%%%%%%%

\mainmatter
\setcounter{page}{1}
\pagestyle{fancy}

\chapter{Uvod}
Avtomatsko testiranje iger postaja vedno težje in težje.

Moderne igre postajajo vedno bolj in bolj kompleksne, z različnimi mehanikami, opcijami za zakljulček igre, ogromni odprtimi svetovi, kompleksnim obnašanjem nasprotnikov in še bi lahko naštevali. 
Vse te naštete lastnosti same po sebi odpirajo problem, kako sistematsko pristopiti k testiranju in zagotavljanju kakovosti igre. Saj dandanes več kot prevečkrat vidimo, da igre pridejo do uporabnikov/igralcev z napakami, ki lahko uničijo njihovo izkušnjo.

Namen te diplomske naloge je predstaviti način kako uporabiti strojno učenje za lažje zagotavljanje kakovosti iger in biti prepričan, da so bili vsi aspekti igre dobro pretestirani in so pripravljeni za izdajo na trg.
Poleg predstavitve načina bi rad tudi pokazal primerjavo med ročnim testiranjem ter strojnim učenjem in tudi med avtomatskimi testi ter strojnim učenjem. Menim, da bi se v praksi moj način uporabljal v kombinaciji z temi že bolj uveljanjenimi metodami.

V 1. poglavju bom predstavil ročno testiranje 

V 2. poglavju bom prestavil Avtomatsko tesiranje in razvijanje vodeno z testi (ang. test driven behavior). Avtomatsko testiranje bomo razdelili na teste enot (Unit teste), integracijske teste (Integration test). Razložiti pojma test driven development in behavior driven development

V 3. poglavju bom predstavil karakteristike igre in okolje v katerem bomo testirali našo igro. 

V 4. poglavju bom prestavil testiranje igre z strojnim učenjem in opisal korake testiranja. 

v 5. poglavju bom primerjal rezultate in predlagal praktično uporabo te metode.